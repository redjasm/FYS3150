\documentclass[english,notitlepage]{revtex4-1}  % defines the basic parameters of the document
%For preview: skriv i terminal: latexmk -pdf -pvc filnavn



% if you want a single-column, remove reprint

% allows special characters (including æøå)
\usepackage[utf8]{inputenc}
%\usepackage[english]{babel}

%% note that you may need to download some of these packages manually, it depends on your setup.
%% I recommend downloading TeXMaker, because it includes a large library of the most common packages.

\usepackage{physics,amssymb}  % mathematical symbols (physics imports amsmath)
\include{amsmath}
\usepackage{graphicx}         % include graphics such as plots
\usepackage{xcolor}           % set colors
\usepackage{hyperref}         % automagic cross-referencing (this is GODLIKE)
\usepackage{listings}         % display code
\usepackage{subfigure}        % imports a lot of cool and useful figure commands
\usepackage{float}
%\usepackage[section]{placeins}
\usepackage{algorithm}
\usepackage[noend]{algpseudocode}
\usepackage{subfigure}
\usepackage{tikz}
\usetikzlibrary{quantikz}
% defines the color of hyperref objects
% Blending two colors:  blue!80!black  =  80% blue and 20% black
\hypersetup{ % this is just my personal choice, feel free to change things
    colorlinks,
    linkcolor={red!50!black},
    citecolor={blue!50!black},
    urlcolor={blue!80!black}}

%% Defines the style of the programming listing
%% This is actually my personal template, go ahead and change stuff if you want



%% USEFUL LINKS:
%%
%%   UiO LaTeX guides:        https://www.mn.uio.no/ifi/tjenester/it/hjelp/latex/
%%   mathematics:             https://en.wikibooks.org/wiki/LaTeX/Mathematics

%%   PHYSICS !                https://mirror.hmc.edu/ctan/macros/latex/contrib/physics/physics.pdf

%%   the basics of Tikz:       https://en.wikibooks.org/wiki/LaTeX/PGF/Tikz
%%   all the colors!:          https://en.wikibooks.org/wiki/LaTeX/Colors
%%   how to draw tables:       https://en.wikibooks.org/wiki/LaTeX/Tables
%%   code listing styles:      https://en.wikibooks.org/wiki/LaTeX/Source_Code_Listings
%%   \includegraphics          https://en.wikibooks.org/wiki/LaTeX/Importing_Graphics
%%   learn more about figures  https://en.wikibooks.org/wiki/LaTeX/Floats,_Figures_and_Captions
%%   automagic bibliography:   https://en.wikibooks.org/wiki/LaTeX/Bibliography_Management  (this one is kinda difficult the first time)
%%   REVTeX Guide:             http://www.physics.csbsju.edu/370/papers/Journal_Style_Manuals/auguide4-1.pdf
%%
%%   (this document is of class "revtex4-1", the REVTeX Guide explains how the class works)


%% CREATING THE .pdf FILE USING LINUX IN THE TERMINAL
%%
%% [terminal]$ pdflatex template.tex
%%
%% Run the command twice, always.
%% If you want to use \footnote, you need to run these commands (IN THIS SPECIFIC ORDER)
%%
%% [terminal]$ pdflatex template.tex
%% [terminal]$ bibtex template
%% [terminal]$ pdflatex template.tex
%% [terminal]$ pdflatex template.tex
%%
%% Don't ask me why, I don't know.

\begin{document}

\title{Project 1 - FYS3150}               % self-explanatory
\author{Josef Ayman M.}                      % self-explanatory
\date{\today}                             % self-explanatory
\noaffiliation                            % ignore this, but keep it.


\maketitle 
    
\textit{List a link to your github repository here!}
    
\section*{Problem 1}
We check that $u(x) = 1 - (1 - e^{-10}) x - e^{-10 x}$ is an exact solution to our Poisson equation.

We simplify it:

\[
u(x) = 1-x+e^{-10}x-e^{-10x}
\]

Then we try to find the second derivative:

\[
u'(x)=-1+e^{-10}+10e^{-10x} \\
u''(x) =-100e^{-10x}
\]


Which when fits the source term $-(-100e^{-10x}) = f(x) = 100e^{-10x}$

And the boundary conditions:

\[
u(0)=1-0+0-1=0 \\
u(1)=1-1+e^{-10}-e^{-10}=0
\]

\section*{Problem 2}
In problem 2 we will write a program that:

\begin{itemize}
    \item Defines a vector of $x$ values.
    \item Evaluates the exact solution at each point.
    \item Writes the results to a file.
    \item It's recommended to use the \textit{\href{https://arma.sourceforge.net/}{armadillo}} library.
\end{itemize}

We can include figures using the \texttt{figure} environment. Whenever we include a figure or table, we \textit{must} make sure to actually refer to it in the main text, e.g.\ something like this: ``In figure \ref{fig:rel_err} we show \ldots''. 
\begin{figure}%[h!]
    \centering %Centers the figure
    \includegraphics[scale=0.75]{problem2/poisson_solution_plot.pdf} %Imports the figure.
    \caption{Write a descriptive caption here that explains the content of the figure. Note the font size for the axis labels and ticks --- the size should approximately match the document font size.}
    \label{fig:rel_err}
\end{figure}

\section*{Problem 3}
\textbf{Discretization of the Domain}

\begin{itemize}
    \item The domain $x$ ranges from 0 to 1.
    \item We divide the domain into $N$ equally spaced intervals, where $h$ is the spacing between grid points, $h = \frac{1}{N}$.
    \item Define grid points $x_i$ as:
    $$ x_i = i h \quad \text{for} \; i = 0, 1, 2, \ldots, N $$
\end{itemize}

Let $v_i$ be the approximation to $u(x_i)$.

To approximate the second derivative $\frac{d^2 u}{dx^2}$ at a point $x_i$, we use the \textbf{finite difference method}. The central difference approximation for the second derivative is:

$$
\frac{d^2 u}{dx^2} \approx \frac{u(x_{i+1}) - 2 u(x_i) + u(x_{i-1})}{h^2}
$$

\textbf{Substitute into the Poisson Equation}

$$
-\frac{u(x_{i+1}) - 2 u(x_i) + u(x_{i-1})}{h^2} = f(x_i)
$$

\textbf{Simplify}

Rearrange the equation to solve for the discrete approximation:

$$
v_{i+1} - 2 v_i + v_{i-1} = -h^2 f_i
$$

or

\begin{equation}{}
    -\frac{v_{i+1} - 2 v_i + v_{i-1}}{h^2} = f_i
\end{equation}

\textbf{Boundary Conditions}

In the problem, the boundary conditions are $u(0) = 0$ and $u(1) = 0$. Therefore:

$$
v_0 = 0 \\v_N = 0
$$

Putting it all together, the discretized version of the Poisson equation is:

$$
v_{i+1} - 2 v_i + v_{i-1} = -h^2 \cdot 100 e^{-10 x_i}
$$

for $i = 1, 2, \ldots, N-1$, with boundary conditions:

$$
v_0 = 0 \quad \text{and} \quad v_N = 0
$$

\section*{Problem 4}
To rewrite the discretized Poisson equation as a matrix equation, we start from the discretized second derivative:

$$
-\frac{u_{i+1} - 2u_i + u_{i-1}}{h^2} = f_i
$$

This can be written in matrix form as $\mathbf{A} \vec{v} = \vec{g}$, where  $\mathbf{A}$ looks like:

$$
\mathbf{A} = \begin{bmatrix}
2 & -1 & 0 & 0 \\
-1 & 2 & -1 & 0 \\
0 & -1 & 2 & -1 \\
0 & 0 & -1 & 2
\end{bmatrix}
$$

$g_i$ is our right hand of the equation.

$$
  \begin{bmatrix}        2 & -1 & 0 & 0 \\        -1 & 2 & -1 & 0 \\        0 & -1 & 2 & -1 \\        0 & 0 & -1 & 2    \end{bmatrix} \begin{bmatrix}        v_1\\        v_2\\        v_3\\        v_4    \end{bmatrix} = \begin{bmatrix}        g_1\\        g_2\\        g_3\\        g_4    \end{bmatrix}
$$

$$
   \begin{bmatrix}        2v_1 & -v_2 & 0 & 0 \\        -v_1 & 2v_2 & -v_3 & 0 \\        0 & -v_2 & 2v_3 & -v_4 \\        0 & 0 & -v_3 & 2v_4    \end{bmatrix} = \begin{bmatrix}
g_1 \\
g_2 \\
g_3 \\
g_4
\end{bmatrix}
$$

We now showed that we can rewrite the equation in the matrix form $\mathbf{A} \vec{v} = \vec{g}$

\end{document}